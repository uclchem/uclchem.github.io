\documentclass{llncs}
\usepackage{graphicx}
\usepackage{natbib}
\begin{document}
\title{UCLCHEM User Guide}
\author{J. Holdship}

\maketitle

\section{Introduction}
\textsc{uclchem} is a gas-grain chemical code written in FORTRAN 95. It solves a couple system of ODEs that give the abundance of every species from a user defined set of species and reactions. As well as the user defined reactions it implements optional processes such as thermal and non-thermal desorption and self-shielding of H2 and CO.\par
%
This document is ordered to present the most useful information first. Therefore it starts with a quick guide to running the code, moves on to more advanced points about the code itself and only later presents detailed information on the modules and processes modelled by UCLCHEM. Those who understand chemical modelling, or simply want to try the code can start with the Quick Start section. For a theoretical overview of the model, our release paper Holdship et al. 2017 \url{http://arxiv.org/abs/1705.10677} contains descriptions of the relevant chemistry. There is also an excellent introduction to astrochemical modelling by Walsh \& Millar \url{http://udfa.ajmarkwick.net/downloads/rate13\_code.pdf}.

\section{Quick Start Guide}
\label{sec:quick}
%
\subsection{Prepare your network}
UCLCHEM requires code files for the network and ODEs to be solved, these are created by Makerates from csv files. Enter the Makerates/ directory and run Makerates.py, then copy everything from outputFiles/ to ../src. If you run Makerates without changing anything, you will get a basic network we use to create the outputs in the examples folder.
%
\subsection{Setup and Compiling}
In src/, open the makefile and check the compiler path at the top refers to your preferred compiler. Add any preferred flags here too and select a physics module by changing the variable ``physics''. The dafault is cloud.f90 and is used for the examples, run make.
%
\subsection{Running the Models}
Make produces an executable which will run the model with the parameter values from defaultparameters.f90. It can also be given an input file as an argument. Run ``./uclchem examples/phase1.inp'' to run the example. Try every .inp file in examples/ and then run the plotting script ``scripts/plotexamples.py''. This should produce plots in examples/test-output which can be compared to those in examples/example-output
%
%
\section{Using UCLCHEM}
\subsection{Inputs - Three ways}
\subsubsection{By Code}
All major inputs for the code can be found in defaultparameters.f90, this file contains explanations of each variable in the comments and has a default value. Running the UCLCHEM executable will run the model with these parameters. If they are changed, run make to relink the modules otherwise nothing will happen.
%
\subsubsection{Input Files}
Any parameter in defaultparameters.f90 can be list in a plain text file that is passed to the executable as an argument. Separate the parameter name and value by a single space and put one parameter on each line. Check examples/ directory for sample input files. To execute with an input file run ``./uclchem path/to/filename''
%
\subsubsection{Python}
The makefile accepts the command ``make python'' which will use wrap.f90 to build a python module. The subroutines in wrap.f90 will each be functions in the module and uclchem can therefore be run from python for the purposes of running grids etc. The user will need to set the inputs of their subroutines in wrap.f90 to the variables they wish to vary.\par
%
This python wrapping uses f2py which can suffer from issues depending on the system. We have found that the combination of f2py and ifort does not work on macOS and recommend mac users use gfortran. We caution users to test their outputs and check for things such as sublimation failing to happen in later model runs. 
%
\subsection{Outputs}
\textsc{Uclchem} has two major outputs. The first is the "full" output including the major gas properties and the fractional abundance of every single species in the network. This is written at every timestep. Optionally, a more machine readable output can be created. If a list of species is given, a columnated file of time, density and temperature along with the abundances of those species will be written. For those using the full output, a set of python functions to read and plot the abundances from the full output file are provided in the scripts/ directory.\par
%
Alternatively, f2py accepts arrays with property intent=out which will mean the subroutine returns the array. Thus, python functions can be created the run uclchem and return the abundances at a specific time or density.
%
\subsection{Phases}
Initial abundances are required for the model. One could take these from observations of dark clouds or invent them. Our solution is to use a two phase approach. In the first phase, the entire gas is assumed to be atomic and we model a cloud collapsing from from a density of 100/cm3 until the density reaches the required value for the science model. The collapse stops at this density and the code either ends or runs until chemical equilibrium is reached. This gives a molecular cloud with abundances that are consistent with the chemical network and the conditions of the cloud. \par
%
The second phase is then the actual model being considered and we start from the abundances obtained in phase 1. If cloud.f90 is used, phase 2 represents a hot core and the gas is heated to some maximum temperature, causing the ice mantles to sublimate. In cshock.f90, the gas is instead subjected to the density and temperature changes caused by the passing shock.\par
%
To run a phase 1 model, set phase=1,readAbunds=0 and abundFile to an output file that can be used to start the next model. The parameter ``phase'' in most physics modules will turn off the physics that the module represents and run a simple cloud instead. readAbunds determines whether abundFile is read to get starting abundances (readAbunds=1) or atomic elements are assumed (readAbunds=0). If readAbunds=0 and abundFile is set, then the final timestep of the model will be written to abundFile so it can be used as a starting point for other models.  
%
\section{Making a Network}
\label{sec:network}
Makerates is a python script that will produce network.f90 and odes.f90 as well as species.csv and reactions.csv. The f90 files provide the necessary arrays to describe the network and the code to produce the rate of change of the abundances for the ODEs. These are required for \textsc{uclchem} to run and the code must be recompiled whenever they are updated. The csv files are simply lists of every species and reaction along with important information such as the reaction rate coefficients. These can be inspected by the user or used to make tables. \par
%
The input files for Makerates are in Makerates/inputFiles/ and include a list of species, a gas phase network (generally UMIST 12) and a further list of any user defined reactions. Examples of these are included, the inputFiles folder contains uclspeciesbasic.csv and uclgrainbasic.csv which along with UMIST12. Together these produce a large gas phase network with a minimal surface network. For every species, the grain reaction file should contain at least one freeze out route. The species list should be a list of every species, along with their mass. The other values in the species file are only required for surface species, the most important of which are the binding energy and enthalpy.\par
%
The grain surface network should in the most minimal case contain a freeze out route for every species. The user may wish to define multiple freeze out routes, this allows for processes such as hydrogenation to be accounted for. For example, the freeze out of CO may follow two routes:
\begin{eqnarray}
CO + Freeze \rightarrow \#CO \\
CO + Freeze \rightarrow \#HCO
\end{eqnarray}
 so that some CO immediately hydrogenates. If we assume that 90\% of CO to freezes as itself (\#CO) and 10\% to \#HCO, the alphas should be 0.9 and 0.1 respectively. This will make the overall freeze out rate of CO correct for the model, whilst setting the proportions according to the assumptions.\par
%
More complicated surface chemistry can be obtained in two ways. Firstly, Makerates and \textsc{uclchem} will treat any reaction with two surface reactants exactly the same way as a gas-phase two body reaction. It may be possible to parameterize the rate of a surface reaction using the alpha, beta and gamma values of the Kooji-Arrhenius equation. Secondly, \cite{Quenard2018} describes a modification to \textsc{uclchem} to include a diffusion formalism for the reaction of surface species. These are recognized by the code when a third reactant called either "CHEMDES" or "DIFF" is included in a surface reaction.\par
%
\section{Physics}
\label{sec:physicsmods}
\textsc{uclchem} has grown from an all purpose chemical code used in the UCL astrophysics group with users typically making large edits to the code to model different types of objects. In an effort to make the code more accessible and reusable, \textsc{uclchem} is now split into modules. The chemistry module (chemistry.f90) is the core code, it sets up and solves the ODEs for the chemical abundances and deals with outputs. The physics module controls the gas properties. This is primarily the density, temperature and visual extinction but other variables are accessible from this module. \par
%
The average user will have no need to change chemistry.f90 and will be able to use \textsc{uclchem} by simply selecting the physics module that applies best to the object they wish to model. For more specific purposes, physics-template.f90 is included, which gives the bare skeleton of what must be included if a user wishes to create their own physics module. \textsc{uclchem} is provided with 4 physics modules: cloud.f90 \citep{viti2004}, cshock.f90 \citep{jimenez2008}, turbfrag.f90 \citep{Holdship2015}, collapse.f90 \citep{Priestley2018} as well as hydro.f90. All are described in more detail below.
%
\subsection{Cloud}
The cloud model (cloud.f90) is the `standard' \textsc{uclchem} model. It follows a single parcel of gas or multiple parcel distributed evenly over a line from the centre of a cloud of gas to the edge. It can be used to study gas in steady state or to follow the evolution of the chemistry as a cloud collapses and a star is formed. Cloud is typically run in two phases. \par
In phase 1, the parcel is initialised with an extremely low density and allowed to collapse. The gas starts in purely atomic form. The model is ran until a chosen time or density is reached. Whilst the chemical abundances are written out at every time step, the main result is the abundances at the final time step. These are written into a separate file (default is startabund.dat) and are used as the initial abundances for phase 2. Whilst phase 1 can be used to study steady state gases or look at collapse modes, it's primary purpose is to create a starting point for phase 2 without assuming initial abundances. Instead, phase 1 gives abundances that are consistent with the network.\par
In phase 2, collapse is generally turned off, though this is optional. The temperature follows one of a number of profiles for cores surrounding protostars of different masses. The profile is chosen by setting tempindx and the protostellar masses are listed in cloud.f90.  In this phase, sublimation of the icecs is also introduced. At pre-determined temperatures, desorption events occur. See \citet{viti2004} for more information on the temperature profiles and evaporation events.
%
\subsection{C-shock}
The C-shock module is the combination of \textsc{uclchem} with the C-shock parameterization of \citet{jimenez2008}. Phase 1 can be run in the same way as the cloud model. However, in phase 2 the gas properties are entirely determined by the shock model. As well as controlling the temperature and density, the shock module includes sputtering of the ices. This is done following the treatment in Appendix B of \citet{jimenez2008}. The contents of the ice mantles are ejected in to the gas by sputtering until the gas temperature reaches 130 K. At this point, the entirety of the remaining ices are released into the gas. The C-shock model outputs the temperature, density and neutral and ion velocities of the gas into file 61 (either fort.61 or a named file) as well as the usual \textsc{uclchem} outputs. We encourage the user to read \citet{jimenez2008} for a full technical discussion of the model as well as \citet{viti2011} and \citet{Holdship2016b} for its combination and application with \textsc{uclchem}.
%
\subsection{Collapse}
This is the module produced for the work done in \citet{Priestley2018}. By setting collapse to a value of 2 or higher, different parameterizations of a collapsing core can be accessed. These are described in the comments at the top of the module code. This code is best run using a single point phase 1 where a simple freefall collapse (collapse=1) is done to produce gas of the same density as the initial core centre and a multi-point phase 2 where collapse is set to some higher value. The multi-point is required as the radius of the core (rout) is divided and the chemistry is sampled at each point. The code does take into account the reduction in rout as the core collapses and each parcel migrates inwards with the collapse.
%
\subsection{Turbulent Fragmentation}
This is the module for the work done in \citet{Holdship2015}. In phase 2, the density of the gas is increased according to an isothermal, non-magnetic shock. 
%
\subsection{Hydro}
Hydro.f90 is a simple physics module that reads in temperature, density and time values from a file and interpolates between them. This module can be used to post-process hydrodynamic simulations to explore the chemistry. The user will almost certainly have to alter either hydro.f90 or the output of their simulations to allow for formatting differences. To build up a multi-dimensional picture of the chemistry, the user can produce files containing the time, density and temperature values for multiple positions or tracer particles and run each as separate single-point \textsc{\textsc{uclchem}} instances.

\section{Customization}
We encourage users to create their own physics modules and physics-template.f90 is provided to encourage this. If none of the physics modules included in the repository suit the objects you wish to modle, then physics-template.f90 provides a base from which you can produce a new module. Ultimately, it needs to set the density, temperature and visual extinction of the gas. Anything else is entirely optional. We would like to encourage users who create physics modules to share them in the main repository once they've reached a usable state.

\section{Troubleshooting}
\label{sec:trouble}
\subsubsection{Convergence Failure}
\textsc{uclchem} uses DVODE to integrate the ODEs for the species' abundances. Variables controlling DVODE can be found in paramters.f90, set to values that are optimal for simple cloud models in phase 1. Most DVODE related errors and warnings can be fixed by adjusting these parameters. In particular, abstol and reltol control the integration accuracy. The default values function well for a variety of standard model runs, achieving fast runs without sacrificing accuracy compared to smaller tolerances. However, for larger networks or models where variables change more quickly, lowering these tolerances will improve stability.

\bibliographystyle{plainnat}
\bibliography{manual.bib}

\end{document}
